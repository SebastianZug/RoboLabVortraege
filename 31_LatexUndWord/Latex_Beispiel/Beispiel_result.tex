\documentclass{article}

\usepackage[ngerman]{babel}
\usepackage{graphicx}
\usepackage{hyperref}

\begin{document}

\title{Das Entenhausener Finger-Problem}

\author{Micky Maus\footnote{Entenhausen Universität} \& Donald Duck}

\maketitle

\tableofcontents

\begin{abstract}
    In diesem Paper wird die Frage beantwortet, warum alle Bewohner von 
Entenhausen nur vier Finger an jeder Hand haben.
\end{abstract}

\newpage

\section{Einleitung und Forschungsfrage}

In den 30er Jahre wurden die Figuren von 
\begin{itemize}
    \item Mickey Maus \cite{mickey}
    \item Donald Duck  \cite{mickey}
    \item Goofy \cite{mickey} sw. 
\end{itemize}

von Walt Disney und Ub Iwerks geschaffen. Eine Besonderheit der Figuren ist, 
dass alle nur vier Finger an jeder Hand haben. Die Forschungsfrage, die in 
diesem Paper beantwortet werden soll, ist, warum das so ist.

\section*{Beweisfoto}
\begin{figure}[h!]
    \centering
    \includegraphics[width=0.5\textwidth]{mickey-mouse-vectorportal-11582.jpg} % Platzhalter für Bild
    \caption{"Beweisfoto" von Micky Maus (Quelle. https://vectorportal.com/de/vector/micky-maus-vektorgrafiken.ai/2595)}
    \label{fig:beweisfoto}
\end{figure}

\ref{fig:beweisfoto} zeigt Micky Maus mit vier Fingern. Für alle anderen Bewohner wurd eine Tabelle angelegt und die Fingerzahl erfasst.

\begin{table}[h!]
    \centering
    \begin{tabular}{c|c}
        Bewohner & Fingerzahl \\ \hline
        Micky Maus &  4 \\
        Donald Duck & 4 \\
        Goofy & 4 \\
    \end{tabular}
    \caption{Bewohner von Entenhausen mit Fingerzahlen}
    \label{tab:my_label}
\end{table}

Versuchen wir das Ganze noch mal mathematisch zu greifen:

$$
\sum_1^41=4
$$

\section{Erklärungsansatz}

In den frühen Tagen der Animation war es üblich, Figuren mit weniger Fingern zu 
zeichnen, da es schneller ging und die Bewegungen flüssiger wirken ließ. Diese 
Stilisierung hat sich über die Jahre hinweg fortgesetzt, obwohl Comics heute digital 
erstellt werden.

\section{Referenzen}

\begin{thebibliography}{9}
\bibitem{mickey} \url{https://de.wikipedia.org/wiki/Micky_Maus}
\bibitem{donald} \url{https://de.wikipedia.org/wiki/Donald_Duck}
\bibitem{goofy} \url{https://de.wikipedia.org/wiki/Goofy}
\end{thebibliography}

\end{document}

