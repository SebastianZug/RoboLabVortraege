\documentclass{article}
\begin{document}

Das Entenhausener Finger-Problem

Micky Maus & Donald Duck

Entenhausen Universität

In diesem Paper wird die Frage beantwortet, warum alle Bewohner von 
Entenhausen nur vier Finger an jeder Hand haben.

1. Einleitung und Forschungsfrage}

In den 30er Jahre wurden die Figuren von 
+ Mickey Maus [1]
+ Donald Duck  [2]
+ Goofy [3] usw. 
von Walt Disney und Ub Iwerks geschaffen. Eine Besonderheit der Figuren ist, 
dass alle nur vier Finger an jeder Hand haben. Die Forschungsfrage, die in 
diesem Paper beantwortet werden soll, ist, warum das so ist.

https://vectorportal.com/de/vector/micky-maus-vektorgrafiken.ai/2595

Das Beweisfoto zeigt Micky Maus mit vier Fingern. Für alle anderen Bewohner wurde eine Tabelle angelegt und die Fingerzahl erfasst.

Bewohner | Fingerzahl
Micky Maus | 4
Donald Duck | 4
Goofy | 4 

Versuchen wir das Ganze noch mal mathematisch zu greifen:

1+1+1+1=4

2. Erklärungsansatz

In den frühen Tagen der Animation war es üblich, Figuren mit weniger Fingern zu 
zeichnen, da es schneller ging und die Bewegungen flüssiger wirken ließ. Diese 
Stilisierung hat sich über die Jahre hinweg fortgesetzt, obwohl Comics heute digital 
erstellt werden.

3. Referenzen

[1] https://de.wikipedia.org/wiki/Micky_Maus
[2] https://de.wikipedia.org/wiki/Donald_Duck
[3] https://de.wikipedia.org/wiki/Goofy


\end{document}
